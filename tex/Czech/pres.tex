\documentclass{beamer}

%math
\usepackage{amsmath}
\usepackage{amsfonts}
\usepackage{amssymb}
\usepackage{default}

%theme
\usetheme{AnnArbor}
\usecolortheme{crane}

%footer
\makeatletter
\setbeamertemplate{footline}
{
	\leavevmode%
	\hbox{%
		\begin{beamercolorbox}[wd=.30\paperwidth,ht=2.25ex,dp=1ex,center]{palette primary}%
			\usebeamerfont{author in head/foot}\insertshortauthor
		\end{beamercolorbox}%
		\begin{beamercolorbox}[wd=.40\paperwidth,ht=2.25ex,dp=1ex,center]{palette secondary}%
			\usebeamerfont{title in head/foot}\insertshorttitle
		\end{beamercolorbox}%
		\begin{beamercolorbox}[wd=.30\paperwidth,ht=2.25ex,dp=1ex,right]{palette primary}%
				\insertshortinstitute\hspace*{4ex}\insertframenumber{} / \inserttotalframenumber\hspace*{2ex}
		\end{beamercolorbox}
	}%
	\vskip0pt%
}
\makeatletter

\author{Kochurov Maxim}
\title{Czech Parliament
}
\institute{MSU}
\begin{document}
\begin{frame}
	\maketitle
\end{frame}
\section{Parliament}
\subsection{Chambers}
\begin{frame}
	There are two chambers in Czech parliament
	\begin{itemize}
		\item the House of Parliament
		\item the Senate
	\end{itemize}
Every citizen who is at least 18 years old is entitled to vote for candidates to the House of Parliament and the Senate.
\end{frame}
\section{House of Parliament}
\subsection{Rules, Members}
\begin{frame}
	\begin{itemize}
		\item The House is traditionally filled with 200 members. They are elected \textbf{once 4 years}
		\item Proportional representation is basis principle for election
		\item Minimum age required to be elected is 21. You must be a Czech citizen
		\item Members are elected as representatives of individual political parties
	\end{itemize} 
\end{frame}
\subsection{Responsibilities}
\begin{frame}
	Responsibilities of HoR are common
	\begin{itemize}
		\item Discusses and approves laws
		\item A member or group of members is entitled to draft bills
		\item Decides on the appearance of the state budget
		\item Is entitled to declare lack of confidence in the Government
		\item Elects the President of the Republic during a joint meeting with senators
	\end{itemize}
\end{frame}
\begin{frame}
	Members are also divided into 18 Committees of the House of Parliament. Scope of their responsibility varies and can be the following
	\begin{itemize}
		\item Budget
		\item Aricultural
		\item Foreign relations
	\end{itemize}
	The committees should perform the function of expert guarantor for a specific field of social and political life within the country.
\end{frame}
\section{Senate}
\subsection{Rules, Members}
\begin{frame}
	Most interesting thing is rotation in Senate. This can prevent long term coalitions there.
\begin{itemize}
	\item Elected in a cyclic way for 6 years every 2 years. So 1/3 part is rotated every 2 years
	\item 40 years is minimum age for a citizen to be a member
\end{itemize}
\end{frame}
\subsection{Responsibilities}
\begin{frame}
	\begin{itemize}
		\item Discusses and approves drafts of bills, which it receives from the House of Parliament, and proposes as well
		\item Expresses its (dis)agreement with international treaties
		\item Elects the President of the Republic at a joint meeting of both chambers
		\item Only the Senate may file an action with the Constitutional Court against the President of the Republic for treason
		\item The Senate is not authorised to make any decisions regarding the Czech Republic budget
	\end{itemize}
\end{frame}
\section{President}
\subsection{Overview}
\begin{frame}
	The President of the Republic is the head of state and the supreme commander of the armed forces.
	\begin{itemize}
		\item Elected by all Czech citizens above the age of 18
		\item A person can be elected to the office no more than twice
		\item A President is elected once every five years
	\end{itemize}
\end{frame}
\section{References}
\begin{frame}[t]
	Resources Used
	\begin{itemize}
		 \item Czech Republic Political System -- \href{http://www.czech.cz/en/88070-czech-republic-political-system}{http://www.czech.cz/en/88070-czech-republic-political-system}
		 \item Github Repo -- \href{https://github.com/ferrine/parliament}{https://github.com/ferrine/parliament}
		 \item Data Resources -- \href{https://en.wikipedia.org/wiki/Czech_legislative_election,_2017}{https://en.wikipedia.org/wiki/Czech\_legislative\_election,\_2017}
	\end{itemize}

\end{frame}
\end{document}
